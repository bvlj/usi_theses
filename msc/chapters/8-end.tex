%%%%%%%%%%%%%%%%%%%%%%%%%%%%%%%%%%%%%%%%%%%%%%%%%%%%%%%%%%%%%%%%%%%%%%%%%%%%%%%%
\chapter{Conclusion}

\begin{chapterBody}

In this thesis, we presented a number of contributions to the Expression
Tutor platform that enable the automatic generation and assessment of 
personalized learning activities for each student at scale.

% P1 - Identify Expressions
We started by designing and implementing a way for instructors to specify what
learning goal the automatically generated activity should cover. This is
accomplished through the use of the Expression Tree Language
(\textbf{Chapter~\ref{sec:etl}}). This domain–specific language allows
instructors to  select expressions in sources code files that pertain to their
learning goals.
As seen in the evaluation of the Expression Tree Language
(Section~\ref{sec:etl-eval}), it is capable of representing all
expression–related constructs of a complex ``real world'' programming language
such the Java programming language (version 11).
We use this query to search for specific expressions that pertain to the
instructor's learning goals in the coding assignment of students.

% P2 - Distractor Nodes + Activity Generator
We use the selected expression to generate a personalized Expression Tutor
activity for students. To do so, we automatically generate distractor nodes
(\textbf{Section~\ref{sec:dn-distractors}}).
The criterion used to generate these nodes is to attempt to re–create the
incorrect nodes that students drew when building expression trees in exams
(Section~\ref{sec:dn-distractors-gen}).
Then, we implement a system to automatically generate Expression Tutor
activities from the each student's repositories on GitHub by using a GitHub
Action (\textbf{Section~\ref{sec:impl-aag}}).

% P3 - Assessment + Student Feedback + Instructor Feedback
Upon submission of the student's activity, the system performs an
automatic assessment  \hfill\break
(\textbf{Section~\ref{sec:fb-assess}}) and provides formative
feedback to the student (\textbf{Section~\ref{sec:fb-ff}}).
This feedback informs the students about the mistakes in their
submission and guides them towards the correct one by providing hints.
The same assessment system also gathers all assessments of the activities of
the different students and provides instructors with feedback 
(\textbf{Section~\ref{sec:fb-sf}}) that offers both aggregate information about
the mistakes found in students' submissions and the possibility to view each
individual submission in detail.

\section{Future Work \& Limitations}

In this section, we detail possible enhancements to our system that would
help overcome existing limitations.

\subsection*{Expanding Query Capabilities of ETL}

One area in which the capabilities of ETL could be improved is support for more
advanced type querying. At the time of writing, the types are compared as
\textit{plain strings} by the query DSL. By introducing the concept of type
analysis to ETL, it would be possible to write queries with generic types.
One such example would be a query that involves a sub–expression of type 
``array of an undetermined type \texttt{T}'' and another sub–expression of
type \texttt{T}. The main challenge of implementing such a feature lies in how
different languages have different (and sometimes even unsound) type
systems~\cite{lu_gradual_2022}.
It would require additional information generated while converting expressions
from the source code that defines a language–agnostic representation of a type
hierarchy of all types involved in said expression.

\subsection*{Improvements to Distractor Node Generation}

The distractor node generators that have been implemented and detailed in this
document were designed by \textit{just} looking at the answers found in two
different sets of exams submissions. The expressions students were tasked to draw
an Expression Tree for were focusing on particular learning goals, and thus, the
incorrect nodes that were found by manually inspecting the exams were always
going to be biased towards said learning goals. To make the selection of
distractor nodes that are generated better, it would be necessary to look at
even more exams, possibly those that tasked students to draw Expression Trees
for expressions that were selected with different learning goals in mind.

Moreover, it would be possible to improve the current generators by applying
additional heuristics to exclude certain distractors that are less likely to be
picked by students. Reducing the number of distractors by removing those that
are \textit{too easy to spot} also reduces the cognitive load on the student who
has to look for the correct node among all those provided in the initial state
of the activity.
The assessment system and activity groups that were introduced in this document
are features that will be useful in easily gathering such data to improve the
quality of the automatically generated distractors in the future.

\subsection*{Support for Languages Other Than Java in the GitHub Action}

The GitHub Action (and thus the associated Docker image) allows to 
automatically generate activities from the code in the student's repository.
The current implementation of said action only includes the components necessary
to perform source code analysis and expression extraction for code written in
the Java programming language. It would definitively be useful to also include
support for extracting expressions from the other languages already supported by
the Expression Tutor platform such as Python. No implementation detail or code
change would be necessary other than just installing the necessary additional 
components to the Docker image, as the expression service code is already
capable of autonomously invoking them should they be available.

\subsection*{Evaluation of Feedback Functionality}

The feedback system has not yet been properly fully evaluated to verify whether
all requirements were met. In particular, there is some initial evaluation of
the formative feedback provided to students, but follow–up interviews would be
necessary. Moreover, there has been no evaluation of the feedback
provided to instructors.

\subsection*{Support for Additional Types of Activities}

At the time of writing, the system supports only one type of Expression
Tutor activities. It would be possible to implement different types of
activities, each with a different configuration for the initial state and
assessment process. One such example would be \textit{typing} activities.
In such cases, instead of being provided nodes that need to be selected and 
connected to form an expression tree, students would receive an already
composed expression tree. The task consists of labeling all the nodes with the
correct type and identifying possible type errors. The assessment process would
differ because now the correctness of the node contents would no longer be
relevant.

\end{chapterBody}
