%%%%%%%%%%%%%%%%%%%%%%%%%%%%%%%%%%%%%%%%%%%%%%%%%%%%%%%%%%%%%%%%%%%%%%%%%%%%%%%%
\begin{abstract}

\addchaptertocentry{\abstractname}

%%% Problem/Motivation
Learning how to program is hard.
In particular, code comprehension has been shown to be 
challenging and important for a positive learning outcome.
Students don't always understand the code they write.
This has been exacerbated by the advent of large language models
that automatically generate code that may or may not be correct.
Now students don't just have to understand their own code,
but they have to be able to critically analyze automatically generated code.

Prior research has shown that while expressions are one of the most 
prevalent constructs in student code,
they are one of the most neglected constructs in programming education.
Even in imperative languages like Java, 
expressions make up a significant fraction of the code students produce.

%%% Solution
To support teaching and assessing students about expressions,
we utilize a notional machine called ``Expressions as Trees''.
Notional machines are pedagogical devices that assist the understanding
of certain aspects of programming.
The "Expression as Tree" notional machine focuses on expressions:
students have to decompose source code expressions
and represent them as trees.

In this document, we define and implement an assessment platform around
this "Expression as Trees" notional machine.
Our platform can automatically generate personalized learning activities
for students based on students' own source code.
Our platform asks students to construct expression trees
for interesting expressions in the code they submitted in programming assignments.
It then automatically assesses the students' expression trees and
provides feedback to both students and instructors.

%%% Evaluation
We evaluate the various aspects of the implemented system, including the
automatic activity generation and the formative feedback provided to students.
The evaluation highlights the usefulness of the automatic system for a
positive learning experience that can align with the instructor's learning
objectives.

\end{abstract}
